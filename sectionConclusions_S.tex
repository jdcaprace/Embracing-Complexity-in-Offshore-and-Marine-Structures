\section{Conclusions}
Systematic and objective analysis of complexity in ship and offshore industry is important for several reasons. First, it helps design engineers to develop a better understanding of various aspects of complexity and thereby evolve toward simpler design solutions. Second, it enables design automation tools to systematically assess different design alternatives based on their inherent complexities.


These methodologies will provide:
\begin{itemize}
\item an aid for designers and managers in order to compare various design alternatives on the basis of complexity,
\item an environment which supports strategic decisions made as early as possible to make ship and offshore structures more cost-effective,
\item a monitoring of the sources of complexity which helps to determine the consequences of decision making early on during the design process,
\item a spotting of the sources of complexity which helps to reduce "design effort", that is, shortening production time and cutting project costs.
\end{itemize}


Fundamentally, these methods will provide design engineers with objective, quantifiable measures of complexity, aiding rational design decision making.


The measures proposed are objective as they are dependent not on an engineer's interpretation of information, but rather on the model generated to represent the ship and offshore structures. This objectivity is essential to using the complexity measures in design automation systems. A prospective computer-aided system should also be capable of assisting innovative design. It should not just provide a limited series of conventional solutions. To this end, design engineers should be provided with well-defined and unambiguous metrics for the measurement of different types of complexities in engineered artefacts. Such metrics aid designers and design automation tools in objective and quantitative comparisons of alternative design solutions, cost estimation, as well as design optimization.
