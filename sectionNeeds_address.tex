\section{Needs addressed}


According to \cite{Gaspar12} and \cite{GasparNTNU13} the ship design information is continuously growing in time-line, \cite{Kolmogorov83} states that the more information need an object or process to be defined, the object becomes more complex. While simplicity involves few degrees of freedom, in contrast to vessels behaviors that should be treated as the composition of many parts that interact  of a whole-not so simple. \cite{Alexiou10} notes that a definition of a "complex system" in the literature is not formally defined despite several years of research in the area, then the complexity of a structure depends on how she is described, as a feature, difficult to quantify and define precisely.


Usually many complex systems have a structure that can hardly be decomposed into several parts or hierarchically, on  \cite{Simon62} complexity tends to characterize something with many intricate parts. To characterize the complexity depends on the approach in science where it is applied, related to detailed knowledge and interactions of its parts.  Based in areas such as: engineering, IT technology, management, economy, arithmetic, statistics, data mining, life simulation, psychology, philosophy, information, linguistics, etc., it is clear the huge diversity in the concept of complexity. According to \cite{Bonchev05} is most often defined "complexity" as equivalence and diversity of the elements of the studied system. Is important to emphasize that the definition of being something "complex or not" depends on time.


In \cite{Simon96} a series of observations about particular kinds of complex systems  found in various sciences were recorded, then highlights the following characteristics:


\begin{itemize}
\item Redundancy is a common characteristic of most of the complex systems found in the world.
\item Complex system description contains much information or many parts.
\item The relationships/interactions between the parts in the system are always present.
\item Complex systems composed of subsystems and components of a large system often can be described in a hierarchical manner; are organized into modules with matching redundant components.
\end{itemize}


\cite{Simon62} says that a "hierarchy system" consists in interconnected subsystems each other, in a structure that gradually reaches a elemental subsystem. In hierarchic systems can be distinguish between: the interactions between and within the subsystems, regardless of the content of this system. Hierarchies have a property called near-decomposability where simplifies the description of a complex system making it easier to understand how you can store the information necessary for the development and reproduction of the system.


The term  "complexity" is used daily, considering some elements are more complex (more difficult to: design, assembly, manufacture, customizing, finished, etc.) than others. But this measure of complexity is subjective and is not enough for the engineering analysis.


According to \cite{rodriguez2002} throughout the years "complexity" caught the attention of researchers in all fields of science (social sciences and engineering) and their investigations have different definitions of this term in their quest to understand it. The need to define and measure the complexity is that nowadays the systems, technologies, machinery, etc. have a overwhelming complexity and required best techniques/ methods to resolve those into a more comprehensible form, eg. in \cite{rodriguez2002} developed methods to quantify the complexity of a product to evaluate the manufacturing analysis.


The shipbuilding is a worldwide industry, dominated by industrialized countries like South Korea, Japan and China. In this highly competitive sector, innovation is a key factor for success. Besides building highly complex structures, such as LNG, LPG, drilling ships, semi-submersible platforms, FPSOs, OSVs, PSVs, AHTSs, etc. Brazilian shipyards are forced to increase their manufacturing efficiency in order to become competitive with low labour cost countries. A complexity metric is a way to reach this objective.


To solve the problems in engineering and related management's the tendency of researchers, including \cite{Chryssolouris94}, \cite{Little97}, and \cite{Calinescu00} is recognizing the relevance and implement measurement complexity objectively. To measure complexity empirical measures are used by industry. The spread of possible measures is a problem, \cite{ElMaraghy12} and \cite{Milner13}: The large numbers of metrics available (number of items in the ship, analysis of production sequence and assemblies, etc.) induces problems. The uncertainties are: How do you know you are using the most appropriate ones?, How do you know you have enough accuracy?, How can you tell if the complexity has decreased globally or only in some measures?
