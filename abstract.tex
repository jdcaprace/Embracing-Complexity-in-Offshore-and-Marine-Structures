\begin{abstract}
%Contextualization
The complexity of large engineering projects such as ships and offshore structures has constantly increased in the past decades.
%Gap - Altough ..., However ...
There is a need to manage and reduce complexity of such projects in order to avoid risks, reduce costs and make robust and functional products. This is particularly important in large scale projects, where small changes may propagate disastrous effects.
%Purpose - This paper propose ... describes
%añadido:
This paper describes  a complexity metric for ships and offshore structures. The aim is to provide the designer and managers a strong tool which allow real-time monitoring of the vessel future performance, for designers to evaluate different design alternatives and choose the best option.
%******RETIRADO*****
%This paper introduces a complexity metric for ships and offshore structures. The goal is to provide the designers and managers with such information throughout the design process so that an efficient design is obtained at the first design run. Real-time assessment of complexity and quality measurements is rather imperative to ensure efficient and effective optimality search, and to allow real-time adjustment of requirements during the design. %REVISAR**********************
%Methodology - .... results in ....
The overall design complexity was considered as a combination of the compactness complexity, the assembly complexity, the material complexity and the shape complexity for both steel structures and outfitting pipes.
%Results - We suggest that ..., The results point to the development of ..., These findings provide ...
Results are presented based on a Handling Tug Supply boat (AHTS).
%Conclusions - We suggest that the
%añadido:
We suggest that the new method is effective in helping design engineers to understand the different aspects of the complexity and evolve into solutions of simple design, helping to the decision process for the design of ships.
%******RETIRADO*****
%We suggest that the new method is effective in giving a complementary aid to decision process for ship designers %REVISAR*******************


\keywords{Design complexity \and Shipbuilding \and Optimization \and Offshore}
% \PACS{PACS code1 \and PACS code2 \and more}
% \subclass{MSC code1 \and MSC code2 \and more}
\end{abstract}