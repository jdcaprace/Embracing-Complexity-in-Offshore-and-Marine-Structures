\section{Introduction}
"Simplicity is the ultimate sophistication", attributed to Leonardo da Vinci (1452-1519).


"Simplicity is the soul of efficiency", Richard Austin Freeman (1862-1943)


"If you can't explain it to a six year old, you don't understand it yourself", attributed to Albert Einstein (1879-1955).


These quotes are showing how the simplicity, i.e. the opposite of complexity, is seen along the past centuries. Simple is beautiful, complex is not. 
More specialized systems lose this simplicity, although complex systems will evolve from simple systems to be efficient.
Often, small and simple is more powerful than big and complex. Racing into complexity is rarely the solution. We must be able to see the bigger picture, to focus on what really counts, what really brings results and what your real priority should be. Only then, once you know this, we can make things simpler.


\cite{Davies10} made a comparison where the frequency of the words of a database of over five million books were analyzed, although not all articles were related to engineering, reflect what society has written the last century. Fig. \ref{FigureComplexityTerm} presents a plot from this database with the frequency of words "complex", "complexity" and "complexes" in the English Corpus from the period 1810 through 2012. Demonstrates that the frequency of these words is constantly increasing since the industrial revolution.
