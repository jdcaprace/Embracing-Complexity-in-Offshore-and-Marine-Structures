\section{Objectives}
Due to the  the Life Cycle Costs (LCC) has some components (capital costs, operational costs, and disposal costs) when complexity of ships and offshore structures increases this LCC also increases. A complex ship have a lengthy, complicated, and expensive design process. Moreover, a complex ship has different levels that are cumbersome as:  
\begin{itemize}
\item Supply chain which various management and logistic problems, 
\item Interconnection of components and sub-assemblies,  
\item Manufacturing using complex process plans with sophisticated manufacturing tools and technologies, and
\item Serviceability is a challenging issue due to existence of many failure modes with multiple effects having varying levels of predictability.
\end{itemize}


Therefore, to eliminate unprofitable details is beneficial objectively measure the complexity of design of ships and offshore structures. Using complexity measures help the designer in creating a final product balanced between cost, manufacturing and assembly difficulties. In the field of manufacturing processes ships "complexity" is vital to achieve an optimal design in specialized vessels. However progress in the area of complexity metrics have been few, and have not been used usefully.  \cite{Tang01} says "complexity must be explored further with industry experiments" and showing that only 20\% of the studies in the area attempt to quantify the problem. In his work shows the steps for calculating the complexity of systems to study the behavior of the complexity function.


Through this work will be defined measures to be used in conjunction with others such as assessment of production, operation, logistics and, maintenance efficiency, looking develop the means to quantify the complexity of ships and offshore structures. The definition of Complexity not be performed in a measurable mode by the authors cited here, so further research is essential to enhance the concept of complexity to be implemented at the level of the shipbuilding industry.


The main purpose in this paper is the integration of  ship design model with the complexity evaluation in all the aspects of the  design process, including the  conception, the alternatives, assembly and production problems, and others parameters can be detected in the early stage of the design process. For the designer an innovation in this study is to have a strong methodology with efficient models, allowing real-time monitoring of the evolution of the vessel enabling evaluate different design options to decide which is the best alternative.
