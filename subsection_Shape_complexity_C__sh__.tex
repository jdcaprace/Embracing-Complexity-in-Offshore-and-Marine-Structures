\subsection{Shape complexity -- $C_{sh}$}
The construction of ships obviously involves a large number of steel plates and shapes which form the hull surface panels. These plates and shapes need to be formed so that the hull shape can be developed.


The forming complexity depends largely of the curvature of the plates, \cite{parsons1999scalar}. Two parameters have been used to classify the forming of the hull plates: the Gaussian curvature $K$ and the ratio between the two principal curvatures $R$. The Gaussian curvature is defined as the product of the two principal curvatures (see equation \ref{eq_gaussiancurv}) while $R$ is defined as the ratio between the two principal curvatures (see equation \ref{eq_ratcurv}).
	
\begin{equation}
\label{eq_gaussiancurv}
K = k_1 \times k_2
\end{equation}
	
\begin{equation}
\label{eq_ratcurv}
R = \frac{k_1}{k_2}
\end{equation}
	
Firstly, if the Gaussian curvature is positive, which means $k_1$ and $k_2 > 0$ or $k_1$ and $k_2 < 0$, the shape of the surface is either convex or concave. Where $K$ is zero, the surface is ruled, developable or planar. In a planar surface, both $k_1$ and $k_2$ are zero; while in a ruled surface, either $k_1$ or $k_2$ is zero. Where $K$ is negative, the shape of the surface is saddle-shaped involving reverse or opposite curvature in two directions.


Secondly, if the ratio $R$ between the two principal curvatures is low, it means that a double curvature is involved, while when the $R$ is high it means a curvature in only one direction is involved.


The curvature can be evaluated both on the centroid of the plate (for low accuracy measure) or on 225 points (grid of $15 \times 15$) for each hull plate. Later, the average of the values can be evaluated to classify the forming complexity of the plates. Table \ref{tab:compshape} gives the different values of the curvature coefficient in function of the values of the Gaussian curvature $K$ and the ratio between the principal curvatures $R$.


\begin{table}
\caption{Values of the curvature coefficient $c$ for the assessment of the shape complexity}
\label{tab:compshape}
\begin{center}
\begin{tabular}{rcc}
           &   $R$ = High &    $R$ = Low \\
 &  {\it One} & {\it Double} \\
 & {\it direction} & {\it Curvature} \\
  $K$ = High &          2 &          7 \\
$K$ = Moderate &          1 &          5 \\
   $K$ = Low &          0 &          0 \\
$K$ = Moderate reverse &          1 &          8 \\
$K$ = High reverse &          2 &         11 \\
\end{tabular}
\end{center}
\end{table}


Shape complexity is given by equation 9 where $c$ represents the curvature coefficient given in Table \ref{tabcompshape} and $n$ the number of analysed points of each steel plates. It should be noted that the shape complexity only have been assessed for curve plates parts.


\begin{equation}
C_{sh} = \frac{1}{n} \sum_{i=1}^{n} c_i
\end{equation}