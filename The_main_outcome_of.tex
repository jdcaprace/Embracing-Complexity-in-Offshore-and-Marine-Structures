The main outcome of the test case is presented in Fig. \ref{ImgMacroComplexity} where we can see the relative complexities of each ship block at starboard, i.e. the compactness, the assembly, the shape and the material complexity as well as the aggregated total complexity. The darker the block are, the higher the complexity.


By analysing the figures, it is interesting to note that the high complexity is generally located in the bottom part of the ship as well as in the fore and aft part whereas the ship hull has a big curvature. Nevertheless, other areas of the ship do not have uniform complexity. Some blocks are much more complex than others. We can mention here for instance that:
\begin{itemize}
\item The highest compactness complexity appear on the block where the main towing winches are installed
\item The superstructure parts presents a higher compactness complexity mean
\item There is dissymmetry's between starboard and port-side blocks complexity of the ship explained by the installation of specific equipment's such as crane
\item The upper deck blocks are presenting high material complexity which means that standardization could be improved
\item Assembly's complexities as shown various inconsistencies of the assembly planning of the ship. For instance, one specific block presented an assembly complexity 10 times higher that the average (outlier) which should lead to a modification of the design
\end{itemize}


The managers can define an upper and a lower complexity limit for each type of block in order to control the design. Moreover, the composition of the complexity metric with the four factors can orient the designer to revise the appropriate design variables in order to reduce the global complexity of the ship during the design phase. By arranging the structural details and outfitting parts of a ship in a way that enhances the modularity of steel components, standardizing the scantling and simplifying the shape of the components, it is possible to eliminate unnecessary welding's, lengths of piping, ventilation ducting, and many other sources of production and maintenance cost. All of these efforts will result in a reduction of man-hours, material cost and construction time, resulting in a reduction in recurring construction costs.


Experience has shown that structural detailed arrangements that were made during the early stages of design were often carried through detail design without any attempt at optimization. The system deals with the geometric de-tails of the design and highlights the relative complexities of ship sections. It quickly provides measurements of complexity but not yet in real-time. Therefore, it is particularly suitable in design, where fast response to design modifications is quite imperative for the search of optimality.
