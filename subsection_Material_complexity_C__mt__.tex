\subsection{Material complexity - $C_{mt}$}
According \cite{Rigterink2013} the complexity factor of the material is calculated by finding the number of combinations of plate thickness and type of material and the type of stiffener, size and material. 
Given a total approach of the ship, and specifically  the stiffened structure of ship, material complexity has been defined for an assembly by equation \ref{eq_material}.


\begin{itemize}
\item For the plates - $C_{pt}$ - the material complexity is the number of the different combinations between plate thickness and material type. For instance, an assembly containing 10 steel plates of $20 \, mm$, 5 aluminum plates of $20 \, mm$ and 3 steel plates of $15 \, mm$, the complexity will be equal to 3.
\item For the stiffeners - $C_{st}$ - the material complexity is the number of the different combinations between profile types, profile scantling and material types. For instance for an assembly containing 35 steel bulb profiles of $100 \times 6 \, mm$, 10 steel bulb profiles of $100 \times 8 \, mm$ and 5 aluminum bulb pro-files of $100 \times 8 \, mm$, the complexity will be equal to 3.
\item For pipes - $C_{pi}$ - the material complexity is the number of combinations between nominal diameter, pipe thickness and material types.
\end{itemize}


\begin{equation}
\label{eq_material}
C_{mt} = C_{pt}+C_{st}
\end{equation}


	
