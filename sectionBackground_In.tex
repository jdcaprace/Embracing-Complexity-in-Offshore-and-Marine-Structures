\section{Background}
In the past Ship and Offshore design has been treated as an art and not as science, that depended on naval architects experience and other professionals in each of the areas of engineering that should be develop in practice. The design was using heuristic methods deriving from trial and error process, being gradually replaced by knowledge and application of different techniques and computer design programs.


Nowadays is considered the design of ships and offshore platforms as an unstructured processes (ad hoc), since it is the integration of a set of activities such as: design, production, costing, quality control, etc. to get a single final product. Due to the design should be a feasible construction to avoid incurring extra expenses for correction or rework, which also must fulfill  risk, performance, cost, and customer requirements criteria and all applicable regulations: security, stability, structural, environmental, and comfort for passengers and crew, \cite{CapracePRADS10}.


Is the shipbuilding industry the same as other manufacturing industries? The answer is definitely "no". However, many basic management principles hold for the different industries. In a broad sense, the organization is the same. Moreover, the mechanical process in ship construction is not so very different. We can find welding, electrical work, piping, woodwork and painting in other industries like the automotive industry, aeronautical industry as well as in the construction industry. Then how does shipbuilding management differ from these industries or any other "repetitive" manufacturing? The difficulty of building a ship or an offshore structure is significant as it combines small series, short time to market, high complexity, bad working conditions, low standardization, confined space and bad accessibility, iterative design spiral loop, etc. These elements justify why it is not possible to directly apply the recent developments coming from other industries like automotive or aeroplane industry to the shipbuilding industry.


Currently "productivity" is one of the most important attributes in shipbuilding as one of the major concerns in production engineering, and equalling construction efficiency and design efficiency. The concept of Design For Production (DFP) was developed to increase this "productivity" by scientific and shipyards community, in \cite{USA1999} was defined the objective of DFP as follows:
"Design to reduce production costs to a minimum, compatible with the requirements of the vessel to fulfil its operational functions with acceptable safety, reliability and efficiency".


The advantage of DFP is compliance of design and quality requirements, optimizing the manufacturing functions (fabrication, assembly, test, procurement, delivery, service, repair, etc.) reducing the production labour. DPF's aim is inside that production process design decisions are implicit.  Compliance delivery times for commercial vessels affecting production due to the haste in design, material and equipment procurement, and finally construction, resulting in an overlap of phases or extreme use of resources in certain periods of time. Resulting management difficulties if last minute changes are made to the design engineering. It is essential to periodically study the detailed design process and its impact on construction in order to improve the process and its integration with the construction. Correctly use of DFP  can reduce cost and manufacturing time by reducing the complexity of production.


Today, ship designers and shipyards use various CAD-CAM tools to aid in the design and production of ships. Nevertheless, none of these software's are able to give real time information to the designer regarding the efficiency of his design in terms of production, operation, maintenance and life cycle costs. Indeed, cost assessment for ship CAD-CAM software developers is often considered as a tedious and time consuming task which is very specific and different for each shipyard. There is a need to systematically study the detail design process and its impact on construction with the objective to improve the process and its integration along the life periods of the product (construction, operation, disposal). The development of a complexity analysis must be viewed as a alternative to cost assessment which might be similar for all shipyards.


In shipbuilding industry, as in other  heavy industries,  the holistic approach to shape a project is still insufficient  due to product complexity. So the evaluation of complexity must be done from the conception of the construction project considering that the efficient production of a vessel, is the division of it, in the independent elaboration of each block. Each working stages in the main processes of shipbuilding have a different measures for concept design evaluation (e.g. To minimizes rework due to poor quality is necessary "Design for Quality", while to cut assembly time is necessary "Design for Assembly", and to take out the operational inefficiencies is necessary "Design for Operation"). Still remains a need the assessment design concept globally and unified, because is directly related to overall shipbuilding productivity.
