\section{The complexity model}
This paper explores the relationships between several complexity factors for both ship and offshore structures. Developments have been focused on structures (i.e. mainly steel parts) and outfitting (only piping parts and not electrical systems, HVAC, etc.).


A combination of some complexities as: compactness, assembly, material and shape are considered in this paper for a total design complexity:
\begin{itemize}
\item Compactness complexity -– $C_{cp}$ -– Ability to perform the products components manufacture. More complex products need a set of larger pieces, more complicated classifiers and manufacturing effort increases. In general "When exist more simple components in a product, individual parts are simpler". Otherwise "When exist less simple components in a product, individual parts are more complex".
\item Assembly, sequence, process complexity -– $C_{as}$ –- Ability to easily assemble the individual parts of products. More complex products require a larger set of pieces, increasing the effort of assembly. In general "When exist more components in a product, the product is complex to assemble". 
\item Shape complexity –- $C_{sh}$ –- Ability to easily perform bending of plates and stiffeners with complex shapes, including single and double curvatures. More complex products often have more complex shapes making it difficult their production.
\item Material complexity –- $C_{mt}$ –- Ability to use various materials in a product. More complex products require different materials in a set of parts, increasing the production effort. In general "When exist more materials in a product, the product is more complex".
\end{itemize}


The model is given in equation \ref{EqAgregComp}, %where $C_T$ represents the total complexity or aggregated complexity and $w1, \cdots , wi$ represents numerical constants called weighting factors.


\begin{equation}
\label{EqAgregComp}
C_T = \frac{w_1 \cdot C_{cp} + w_2 \cdot C_{as} + w_3 \cdot C_{mt} + w_4 \cdot C_{sh} }{w_1 + w_2 + w_3 + w_4}
\end{equation}


\begin{tabularx}
where		& $C_T$ & is the total complexity\\
			&  & or aggregated complexity,\\
			& $w1, \cdots , wi$	& is numerical constants\\
			&  	&  called weighting factors.
\end{tabularx}


We proposed to calibrate the weighting factors of equation \ref{EqAgregComp} by using the minimization of the linear correlation coefficient between the total complexity and the production time of each ship block. Calculation and validation of the weighting factors were performed on a real passenger ship in \cite{CapraceCAD12}. The results demonstrated the efficiency of the methodology.


Nevertheless, in this paper the results are presented using unitary weighting coefficient because the production times were not available for the presented test case.


\subsection{Compactness complexity –- $C_{cp}$}
The shape complexity, is a numerical amount that represents the degree to which a shape is compact. At times called "shape factor" or "compactness". This study supposed that a steel part has a complex form (non-compact), which is more difficult to manufacture.


In the literature, the researchers used various compactness measures for 2D shapes and 3D solids, \cite{valentan2008}. These classical measurements of shape complexity for 2D shape it relates in large part to the perimeter and surface area, besides for 3D solids relates in large part to the enclosing surface area and volume.


\cite{valentan2008} states that a combination of the existing methods with others currently in development provided an accurate assessment of the complexity shape, and determines appropriate manufacturing procedures regarding to the evaluation.


For 3D shapes, the most usual shape complexity measure is the sphericity (see equation \ref{EqShapeComplexity3D}), is the ratio of the lateral surface of a sphere (with the same volume as the given solid) to the surface area of a 3D solid, defined by \cite{wadell1935volume}. %******RETIRADO*****
This ratio is maximum (=1) for a sphere and minimum (=0) for an infinitely long and narrow shape.


\begin{equation}
\psi = \frac{A_s}{A}= \frac{\pi^{1/3}(6V)^{2/3}}{A}
\label{EqShapeComplexity3D}
\end{equation}


\begin{tabularx}
where		& $\psi$ 	& is the \textit{sphericity},\\
				& $A$		& is the lateral surface of the solid,\\
				& $A_s$	& is the lateral surface of the sphere,\\
				& $V$		& is the volume of the solid.
\end{tabularx}


Finally, it can be calculated the shape complexity $C_{sh}$ for each individual steel component of the ship with the equation \ref{EqShapeComplexityIndFinal}. The average shape complexity of a set of parts such as a ship assembly can be calculated  with the equation \ref{EqShapeComplexityAvgFinal}.


	
\begin{equation}
C_{cp} = 1 - \psi
\label{EqShapeComplexityIndFinal}
\end{equation}


\begin{equation}
C_{cp} = \frac{ \sum_{i=1}^{n} (1-\psi_{n})}{n}
\label{EqShapeComplexityAvgFinal}
\end{equation}


\begin{tabularx}
where		& $C_{cp}$ & is the shape complexity,\\ 
				& $\psi$ 	& is the \textit{sphericity} or the \textit{circularity ratio},\\
				& $n$		& is the number of part inside the assembly.\\
\end{tabularx}


\subsection{Assembly complexity –- $C_{as}$}
The mounting complexity measure in a ship's structure is represented by measuring the level of diversity and interconnectedness of the parts. The greater the variability in the design parameters, becomes the design more complex. Therefore, a modular architecture ship where their subsystems have less functional interdependencies, should have less coupling complexity that a ship with an integral architecture. It notes that high performance is not necessarily a result of the complexity. In brief,  increased interdependence of the various modules and assemblies in the vessel is not necessarily translated into better ship performance.


\cite{ceccatto1988complexity},  and in a recent review of \cite{Shannon01} was defined the method used to establish a quantitative measure of assembly complexity, this research is based on the definition of the complexity of hierarchical systems, the equation \ref{Eq_ComplexityAss} gives the formulation of the assembly complexity.


\begin{equation}
C_{as} = C\left[ \bigcup_{i=1}^n T_i \right] = \sum_{i=1}^{n} C(T_{i}) + N_T \log _2(2^{k_T}-1)
\label{Eq_ComplexityAss}
\end{equation}


\begin{tabularx}
where & $C_{as}$ & is the assembly complexity of a forest composed of $n$ non-isomorphic trees,\\
& $\sum_{i=1}^{n} C(T_{i})$ & is the complexity of the $n$ non-isomorphic sub-trees,\\
& $N_T$ & is the number of elements at the lower level of the tree,\\
& $k_T$ & is the number of branches non-isomorphic.
\end{tabularx}


\subsection{Material complexity -– $C_{mt}$}
According \cite{Rigterink2013} the complexity factor of the material is calculated by finding the number of combinations of plate thickness and type of material and the type of stiffener, size and material. 
Given a total approach of the ship, and specifically  the stiffened structure of ship, material complexity has been defined for an assembly by equation \ref{eq_material}.


\begin{itemize}
\item For the plates -- $C_{pt}$ –- the material complexity is the number of the different combinations between plate thickness and material type. For instance, an assembly containing 10 steel plates of $20 \, mm$, 5 aluminum plates of $20 \, mm$ and 3 steel plates of $15 \, mm$, the complexity will be equal to 3.
\item For the stiffeners -- $C_{st}$ -– the material complexity is the number of the different combinations between profile types, profile scantling and material types. For instance for an assembly containing 35 steel bulb profiles of $100 \times 6 \, mm$, 10 steel bulb profiles of $100 \times 8 \, mm$ and 5 aluminum bulb pro-files of $100 \times 8 \, mm$, the complexity will be equal to 3.
\item For pipes $C_{pi}$ –- the material complexity is the number of combinations between nominal diameter, pipe thickness and material types.
\end{itemize}


\begin{equation}
\label{eq_material}
C_{mt} = C_{pt}+C_{st}
\end{equation}


	
\subsection{Shape complexity –- $C_{sh}$}
The construction of ships obviously involves a large number of steel plates and shapes which form the hull surface panels. These plates and shapes need to be formed so that the hull shape can be developed.


The forming complexity depends largely of the curvature of the plates, \cite{parsons1999scalar}. Two parameters have been used to classify the forming of the hull plates: the Gaussian curvature $K$ and the ratio between the two principal curvatures $R$. The Gaussian curvature is defined as the product of the two principal curvatures (see equation \ref{eq_gaussiancurv}) while $R$ is defined as the ratio between the two principal curvatures (see equation \ref{eq_ratcurv}).
	
\begin{equation}
\label{eq_gaussiancurv}
K = k_1 \times k_2
\end{equation}
	
\begin{equation}
\label{eq_ratcurv}
R = \frac{k_1}{k_2}
\end{equation}
	
Firstly, if the Gaussian curvature is positive, which means $k_1$ and $k_2 > 0$ or $k_1$ and $k_2 < 0$, the shape of the surface is either convex or concave. Where $K$ is zero, the surface is ruled, developable or planar. In a planar surface, both $k_1$ and $k_2$ are zero; while in a ruled surface, either $k_1$ or $k_2$ is zero. Where $K$ is negative, the shape of the surface is saddle-shaped involving reverse or opposite curvature in two directions.


Secondly, if the ratio $R$ between the two principal curvatures is low, it means that a double curvature is involved, while when the $R$ is high it means a curvature in only one direction is involved.


The curvature can be evaluated both on the centroid of the plate (for low accuracy measure) or on 225 points (grid of $15 \times 15$) for each hull plate. Later, the average of the values can be evaluated to classify the forming complexity of the plates. Table \ref{tabcompshape} gives the different values of the curvature coefficient in function of the values of the Gaussian curvature $K$ and the ratio between the principal curvatures $R$.


\begin{table}
\caption{Values of the curvature coefficient $c$ for the assessment of the shape complexity}
\label{tabcompshape}
\begin{center}
\include{tablecompshape}
\end{center}
\end{table}


Shape complexity is given by equation 9 where $c$ represents the curvature coefficient given in Table \ref{tabcompshape} and $n$ the number of analysed points of each steel plates. It should be noted that the shape complexity only have been assessed for curve plates parts.


\begin{equation}
C_{sh} = \frac{1}{n} \sum_{i=1}^{n} c_i
\end{equation}
