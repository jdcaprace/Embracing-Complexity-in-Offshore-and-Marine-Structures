\section{The complexity model}
This paper explores the relationships between several complexity factors for both ship and offshore structures. Developments have been focused on structures (i.e. mainly steel parts) and outfitting (only piping parts and not electrical systems, HVAC, etc.).


A combination of some complexities as: compactness, assembly, material and shape are considered in this paper for a total design complexity:
\begin{itemize}
\item Compactness complexity -- $C_{cp}$ -- Ability to perform the products components manufacture. More complex products need a set of larger pieces, more complicated classifiers and manufacturing effort increases. In general "When exist more simple components in a product, individual parts are simpler". Otherwise "When exist less simple components in a product, individual parts are more complex".
\item Assembly, sequence, process complexity -- $C_{as}$ -- Ability to easily assemble the individual parts of products. More complex products require a larger set of pieces, increasing the effort of assembly. In general "When exist more components in a product, the product is complex to assemble". 
\item Shape complexity -- $C_{sh}$ -- Ability to easily perform bending of plates and stiffeners with complex shapes, including single and double curvatures. More complex products often have more complex shapes making it difficult their production.
\item Material complexity -- $C_{mt}$ -- Ability to use various materials in a product. More complex products require different materials in a set of parts, increasing the production effort. In general "When exist more materials in a product, the product is more complex".
\end{itemize}

The model is given in equation \ref{EqAgregComp}, %where $C_T$ represents the total complexity or aggregated complexity and $w1, \cdots , wi$ represents numerical constants called weighting factors.

\begin{equation}
\label{EqAgregComp}
C_T = \frac{w_1 \cdot C_{cp} + w_2 \cdot C_{as} + w_3 \cdot C_{mt} + w_4 \cdot C_{sh} }{w_1 + w_2 + w_3 + w_4}
\end{equation}

\begin{tabularx}
where		& $C_T$ & is the total complexity\\
			&  & or aggregated complexity,\\
			& $w1, \cdots , wi$	& is numerical constants\\
			&  	&  called weighting factors.
\end{tabularx}

We proposed to calibrate the weighting factors of equation \ref{EqAgregComp} by using the minimization of the linear correlation coefficient between the total complexity and the production time of each ship block. Calculation and validation of the weighting factors were performed on a real passenger ship in \cite{CapraceCAD12}. The results demonstrated the efficiency of the methodology.

Nevertheless, in this paper the results are presented using unitary weighting coefficient because the production times were not available for the presented test case.

