\subsection{Assembly complexity –- $C_{as}$}
The mounting complexity measure in a ship's structure is represented by measuring the level of diversity and interconnectedness of the parts. The greater the variability in the design parameters, becomes the design more complex. Therefore, a modular architecture ship where their subsystems have less functional interdependencies, should have less coupling complexity that a ship with an integral architecture. It notes that high performance is not necessarily a result of the complexity. In brief,  increased interdependence of the various modules and assemblies in the vessel is not necessarily translated into better ship performance.


\cite{ceccatto1988complexity},  and in a recent review of \cite{Shannon01} was defined the method used to establish a quantitative measure of assembly complexity, this research is based on the definition of the complexity of hierarchical systems, the equation \ref{Eq_ComplexityAss} gives the formulation of the assembly complexity.


\begin{equation}
C_{as} = C\left[ \bigcup_{i=1}^n T_i \right] = \sum_{i=1}^{n} C(T_{i}) + N_T \log _2(2^{k_T}-1)
\label{Eq_ComplexityAss}
\end{equation}


\begin{tabularx}
where & $C_{as}$ & is the assembly complexity of a forest composed of $n$ non-isomorphic trees,\\
& $\sum_{i=1}^{n} C(T_{i})$ & is the complexity of the $n$ non-isomorphic sub-trees,\\
& $N_T$ & is the number of elements at the lower level of the tree,\\
& $k_T$ & is the number of branches non-isomorphic.
\end{tabularx}


\subsection{Material complexity -– $C_{mt}$}
According \cite{Rigterink2013} the complexity factor of the material is calculated by finding the number of combinations of plate thickness and type of material and the type of stiffener, size and material. 
Given a total approach of the ship, and specifically  the stiffened structure of ship, material complexity has been defined for an assembly by equation \ref{eq_material}.


\begin{itemize}
\item For the plates -- $C_{pt}$ –- the material complexity is the number of the different combinations between plate thickness and material type. For instance, an assembly containing 10 steel plates of $20 \, mm$, 5 aluminum plates of $20 \, mm$ and 3 steel plates of $15 \, mm$, the complexity will be equal to 3.
\item For the stiffeners -- $C_{st}$ -– the material complexity is the number of the different combinations between profile types, profile scantling and material types. For instance for an assembly containing 35 steel bulb profiles of $100 \times 6 \, mm$, 10 steel bulb profiles of $100 \times 8 \, mm$ and 5 aluminum bulb pro-files of $100 \times 8 \, mm$, the complexity will be equal to 3.
\item For pipes $C_{pi}$ –- the material complexity is the number of combinations between nominal diameter, pipe thickness and material types.
\end{itemize}


\begin{equation}
\label{eq_material}
C_{mt} = C_{pt}+C_{st}
\end{equation}


	
