\subsection{Assembly complexity -- $C_{as}$}
The mounting complexity measure in a ship's structure is represented by measuring the level of diversity and interconnectedness of the parts. The greater the variability in the design parameters, becomes the design more complex. Therefore, a modular architecture ship where their subsystems have less functional interdependencies, should have less coupling complexity that a ship with an integral architecture. It notes that high performance is not necessarily a result of the complexity. In brief,  increased interdependence of the various modules and assemblies in the vessel is not necessarily translated into better ship performance.


\cite{ceccatto1988complexity},  and in a recent review of \cite{Shannon01} was defined the method used to establish a quantitative measure of assembly complexity, this research is based on the definition of the complexity of hierarchical systems, the equation \ref{eqn:Eq_ComplexityAss} gives the formulation of the assembly complexity.


\begin{equation}
C_{as} = C\left[ \bigcup_{i=1}^n T_i \right] = \sum_{i=1}^{n} C(T_{i}) + N_T \log _2(2^{k_T}-1)
\label{eqn:Eq_ComplexityAss}
\end{equation}


\begin{tabular}{lll}
where & $C_{as}$ & is the assembly complexity of a forest composed of $n$ non-isomorphic trees,\\
& $\sum_{i=1}^{n} C(T_{i})$ & is the complexity of the $n$ non-isomorphic sub-trees,\\
& $N_T$ & is the number of elements at the lower level of the tree,\\
& $k_T$ & is the number of branches non-isomorphic.
\end{tabular}


